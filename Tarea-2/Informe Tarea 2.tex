%----------------------------------------------------------------------------------------
%	Paquetes y configuración
%----------------------------------------------------------------------------------------
\documentclass{article}

\usepackage[utf8]{inputenc}
\usepackage[spanish,es-lcroman]{babel}
\usepackage{fancyhdr}
\usepackage{lastpage}
\usepackage{extramarks}
\usepackage[usenames,dvipsnames]{color}
\usepackage{graphicx}
\usepackage{listings}
\usepackage{xparse}
\usepackage{courier}
\usepackage{amsmath}
\usepackage{enumitem}
\usepackage{hyperref}
\usepackage{mathtools}
\usepackage{lipsum}
\usepackage{float}
\usepackage{algorithm}
\usepackage{algpseudocode}
\usepackage{booktabs}
\floatplacement{figure}{H}


% Margenes
\topmargin=-0.45in
\evensidemargin=0in
\oddsidemargin=0in
\textwidth=6.5in
\textheight=9.0in
\headsep=0.25in

\renewcommand{\arraystretch}{1.2}

% Header y footer
\pagestyle{fancy}
\lhead{}
\chead{\tareaRamo\ (\tareaProfesor): \tareaTitulo} % Centro
\rhead{\firstxmark}
\lfoot{UTFSM CSJ - Departamento de Informática}
\cfoot{}
\rfoot{Página\ \thepage\ de\ \protect\pageref{LastPage}} % Pagina
\renewcommand\headrulewidth{0.4pt}
\renewcommand\footrulewidth{0.4pt}

\setlength\parindent{0pt} % Eliminar la indentación

%----------------------------------------------------------------------------------------
%	Inclusión de python con systax highlight
%----------------------------------------------------------------------------------------
\renewcommand\lstlistingname{Script}
\renewcommand\lstlistlistingname{Scripts}

\definecolor{MyDarkGreen}{rgb}{0.0,0.4,0.0}
\lstloadlanguages{Python}
\lstset{
  language=Python,
  frame=single, % Single frame around code
  basicstyle=\small\ttfamily, % Use small true type font
  keywordstyle=[1]\color{BlueViolet}\bf, % Function names
  keywordstyle=[2]\color{Purple}, % Arguments
  keywordstyle=[3]\color{Blue}\underbar,
  identifierstyle=,
  commentstyle=\usefont{T1}{pcr}{m}{sl}\color{MyDarkGreen}\small,
  stringstyle=\color{Purple},
  showstringspaces=false,
  tabsize=5,
  morekeywords={rand,},
  morekeywords=[2]{on, off, interp},
  morekeywords=[3]{y, range},
  morecomment=[l][\color{Blue}]{...},
  numbers=left,
  firstnumber=1,
  numberstyle=\tiny\color{Gray},
  stepnumber=1
}

\newcommand{\python}[2]{
  \begin{itemize}
    \item[]\lstinputlisting[language=Python,caption=#2,label=#1]{#1.py}
  \end{itemize}
}

%----------------------------------------------------------------------------------------
%	Meta Información
%----------------------------------------------------------------------------------------
\newcommand{\tareaTitulo}{Tarea\ 2}
\newcommand{\tareaFecha}{\today}
\newcommand{\tareaRamo}{INF391 Reconocimiento\ de\ Patrones\ en\ Minería\ de\ Datos}
\newcommand{\tareaProfesor}{Marcelo\ Mendoza}

%----------------------------------------------------------------------------------------
%	Título
%----------------------------------------------------------------------------------------
\title{
  \Large\textmd{\textbf{\tareaRamo\\ \tareaTitulo}}\\
  \vspace{0.1in}
  \normalsize
  Universidad Técnica Federico Santa María, Campus San Joaquín\\
  Departamento de Informática\\
  \vspace{0.1in}
  \small{\textsc{\tareaFecha}}\\
  \vspace{0.1in}
  \large{\textsc{Profesor Marcelo Mendoza}}
  \vspace{1.5in}
}

\author{
    \textit{Juan Pablo Escalona} \\
    \small{juan.escalonag@alumnos.usm.cl} \\
    \small{201073515-k}
    \and
    \textit{Rafik Masad} \\
    \small{rafik.masad@alumnos.usm.cl} \\
    \small{201073519-2}
    \and
    \textit{Gianfranco Valentino}\\
    \small{gianfranco.valentino@alumnos.usm.cl}\\
    \small{2860574-9}
}
\date{}

%----------------------------------------------------------------------------------------
% Documento
%----------------------------------------------------------------------------------------
\begin{document}

\maketitle
\newpage

%----------------------------------------------------------------------------------------
% Índice
%----------------------------------------------------------------------------------------
% \setcounter{tocdepth}{3}
% \newpage
% \tableofcontents
% \newpage

%----------------------------------------------------------------------------------------
% Desarrollo
%----------------------------------------------------------------------------------------
\section*{Introducción}
\lipsum

\section*{Análisis de los resultados obtenidos}

\subsection*{Reglas interesantes}

\subsubsection*{El monopolio del pan y pasteles}

En un alto porcentaje de las reglas encontradas, sobre todo las de más alto porcentaje de confianza, encontramos que una combinación de compras implica la compra de pan o pasteles. Lo anterior nos da a inferir que estos productos son altamente consumidos y un porcentaje alto de los consumidores al comprar algún producto llevan además pan o pasteles ya que es algo que es necesario constantemente en el hogar.

\subsubsection*{¿Vegetales?}

En una situación similar pero en menor medida a la de los panes y pasteles existen múltiples relaciones entre distintas combinaciones de compras y vegetales. En particular se repite bastante combinaciones que incluyen instrumentos de cocina. Al igual que los productos anteriores nos da a inferir que estos productos son altamente consumidos y un porcentaje alto de los consumidores al comprar algún producto llevan además vegetales.

\subsubsection*{Desayuno de campeones}

Con una confianza del 87\% podemos determinar que la gente que compra comida para el desayuno, comida congelada y margarina o pañuelos de papel, compran galletas. Lo anterior lo podemos asociar a un comportamiento cultural propio del país de origen de los datos, Estados Unidos, donde la gente que va a comprar comida para el desayuno y congelados, compran galletas.\\

\subsubsection*{Comida de solteros}

Con una confianza del 88\% los que compran galletas y comida preparada también compran comida congelada. Lo anterior lo podemos asociar a que los que compran comida preparada compran por la misma razón comida congelada.\\

\subsubsection*{Asado... y galletas}

Con una confianza del 87\% los que compran galletas, carne y  herramientas para cocinar o vegetales, compran, además, comida congelada. Lo anterior lo podemos asociar a que cuando compran carne y implementos para el asado generalmente lo acompañan con alguna comida congelada fácil de cocinar.

\section*{Conclusiones}

\lipsum

% \section*{Referencias}

%   \begin{itemize}
%     \item ---
%   \end{itemize}

% \section*{Anexo}

  % \python{Tarea 1}{Implementación de los algoritmos}
  % \python{../Codigos/p3}{Pregunta 3. }

\end{document}
