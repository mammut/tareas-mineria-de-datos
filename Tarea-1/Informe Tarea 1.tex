%----------------------------------------------------------------------------------------
%	Paquetes y configuración
%----------------------------------------------------------------------------------------
\documentclass{article}

\usepackage[utf8]{inputenc}
\usepackage[spanish,es-lcroman]{babel}
\usepackage{fancyhdr}
\usepackage{lastpage}
\usepackage{extramarks}
\usepackage[usenames,dvipsnames]{color}
\usepackage{graphicx}
\usepackage{listings}
\usepackage{xparse}
\usepackage{courier}
\usepackage{amsmath}
\usepackage{enumitem}
\usepackage{hyperref}
\usepackage{mathtools}
\usepackage{lipsum}
\usepackage{float}
\usepackage{algorithm}
\usepackage{algpseudocode}
\usepackage{booktabs}
\floatplacement{figure}{H}


% Margenes
\topmargin=-0.45in
\evensidemargin=0in
\oddsidemargin=0in
\textwidth=6.5in
\textheight=9.0in
\headsep=0.25in

\renewcommand{\arraystretch}{1.2}

% Header y footer
\pagestyle{fancy}
\lhead{}
\chead{\tareaRamo\ (\tareaProfesor): \tareaTitulo} % Centro
\rhead{\firstxmark}
\lfoot{UTFSM CSJ - Departamento de Informática}
\cfoot{}
\rfoot{Página\ \thepage\ de\ \protect\pageref{LastPage}} % Pagina
\renewcommand\headrulewidth{0.4pt}
\renewcommand\footrulewidth{0.4pt}

\setlength\parindent{0pt} % Eliminar la indentación

%----------------------------------------------------------------------------------------
%	Inclusión de python con systax highlight
%----------------------------------------------------------------------------------------
\renewcommand\lstlistingname{Script}
\renewcommand\lstlistlistingname{Scripts}

\definecolor{MyDarkGreen}{rgb}{0.0,0.4,0.0}
\lstloadlanguages{Python}
\lstset{
  language=Python,
  frame=single, % Single frame around code
  basicstyle=\small\ttfamily, % Use small true type font
  keywordstyle=[1]\color{BlueViolet}\bf, % Function names
  keywordstyle=[2]\color{Purple}, % Arguments
  keywordstyle=[3]\color{Blue}\underbar,
  identifierstyle=,
  commentstyle=\usefont{T1}{pcr}{m}{sl}\color{MyDarkGreen}\small,
  stringstyle=\color{Purple},
  showstringspaces=false,
  tabsize=5,
  morekeywords={rand,},
  morekeywords=[2]{on, off, interp},
  morekeywords=[3]{y, range},
  morecomment=[l][\color{Blue}]{...},
  numbers=left,
  firstnumber=1,
  numberstyle=\tiny\color{Gray},
  stepnumber=1
}

\newcommand{\python}[2]{
  \begin{itemize}
    \item[]\lstinputlisting[language=Python,caption=#2,label=#1]{#1.py}
  \end{itemize}
}

%----------------------------------------------------------------------------------------
%	Meta Información
%----------------------------------------------------------------------------------------
\newcommand{\tareaTitulo}{Tarea\ 1}
\newcommand{\tareaFecha}{\today}
\newcommand{\tareaRamo}{INF391 Reconocimiento\ de\ Patrones\ en\ Minería\ de\ Datos}
\newcommand{\tareaProfesor}{Marcelo\ Mendoza}

%----------------------------------------------------------------------------------------
%	Título
%----------------------------------------------------------------------------------------
\title{
  \Large\textmd{\textbf{\tareaRamo\\ \tareaTitulo}}\\
  \vspace{0.1in}
  \normalsize
  Universidad Técnica Federico Santa María, Campus San Joaquín\\
  Departamento de Informática\\
  \vspace{0.1in}
  \small{\textsc{\tareaFecha}}\\
  \vspace{0.1in}
  \large{\textsc{Profesor Marcelo Mendoza}}
  \vspace{1.5in}
}

\author{
    \textit{Juan Pablo Escalona} \\
    \small{juan.escalona@alumnos.usm.cl} \\
    \small{201073515-k}
    \and
    \textit{Rafik Masad} \\
    \small{mailrafik@alumnos.usm.cl} \\
    \small{201073519-2}
    \and
    \textit{Gianfranco Valentino}\\
    \small{mailgina@alumnos.usm.cl}\\
    \small{2860574-9}
}
\date{}

%----------------------------------------------------------------------------------------
% Documento
%----------------------------------------------------------------------------------------
\begin{document}

\maketitle
\newpage

%----------------------------------------------------------------------------------------
% Índice
%----------------------------------------------------------------------------------------
% \setcounter{tocdepth}{3}
% \newpage
% \tableofcontents
% \newpage

%----------------------------------------------------------------------------------------
% Desarrollo
%----------------------------------------------------------------------------------------
\section*{Introducción}

En el presente informe se analizan los resultados obtenidos comparando diferentes técnicas de clustering sobre el dataset iris incluido en la librería \textit{sklearn}.

\section*{Análisis de los resultados obtenidos}

\subsection*{1.1. \; k-means}

\begin{description}
  \item[Algoritmo:] k-means
  \item[Parámetros utilizados:] \hfill
    \begin{itemize}
      \item n\_clusters: 3
      \item tol: 0.1
      \item max\_iter: 300
      \item n\_jobs: 1
    \end{itemize}
  \item[Resultados]\hfill
    \begin{itemize}
      \item Errores: 14
      \item Tiempo de ejecución: 0.0123851 [s]
    \end{itemize}
\end{description}

\begin{figure}[H]
  \centering
  \includegraphics[width=0.666\textwidth]{img/K-Means.png}
  \caption{K-means++}
\end{figure}
\newpage



\subsection*{1.2. \; Minibatch k-means}
\begin{description}
  \item[Algoritmo:] Minibatch k-means
  \item[Parámetros utilizados:] \hfill
    \begin{itemize}
      \item n\_clusters: 3
      \item reassignment\_ratio: 0.01
      \item max\_iter: 100
      \item batch\_size: 5
      \item init: 'k-means++'
      \item n\_init: 3
      \item tol: 0.5
    \end{itemize}
  \item[Resultados]\hfill
    \begin{itemize}
      \item Errores: 15
      \item Tiempo de ejecución: 0.016015 [s]
    \end{itemize}
\end{description}

\begin{figure}[H]
  \centering
  \includegraphics[width=0.666\textwidth]{img/MiniBatchK-Means.png}
  \caption{Minibatch k-means}
\end{figure}

En general k-means, al converger a un máximo local, es extremadamente dependiente de las semillas, explicando así las diferencias en la calidad de los resultados por ejecución.
\newpage



\subsection*{1.3. \; HAC}
\begin{description}
  \item[Algoritmo:] HAC
  \item[Parámetros utilizados:] \hfill
    \begin{itemize}
      \item n\_clusters: 3
      \item affinity: euclidean
      \item n\_components: None
      \item linkage: average
    \end{itemize}
  \item[Resultados]\hfill
    \begin{itemize}
      \item Errores: 16
      \item Tiempo de ejecución: 0.002657 [s]
    \end{itemize}
\end{description}

\begin{figure}[H]
  \centering
  \includegraphics[width=0.666\textwidth]{img/HAC.png}
  \caption{HAC}
\end{figure}

\newpage




\subsection*{1.4. \; Ward}
\begin{description}
  \item[Algoritmo:] Ward
  \item[Parámetros utilizados:] \hfill
    \begin{itemize}
      \item n\_clusters: 3
    \end{itemize}
  \item[Resultados]\hfill
    \begin{itemize}
      \item Errores: 16
      \item Tiempo de ejecución: 0.011104 [s]
    \end{itemize}
\end{description}

\begin{figure}[H]
  \centering
  \includegraphics[width=0.666\textwidth]{img/Ward.png}
  \caption{Ward}
\end{figure}

\newpage




\subsection*{1.5. \; DBScan}

\begin{description}
  \item[Algoritmo:] DBScan
  \item[Parámetros utilizados:] \hfill
    \begin{itemize}
      \item min\_samples: 14
      \item eps: 0.5
    \end{itemize}
  \item[Resultados]\hfill
    \begin{itemize}
      \item Errores: 22
      \item Tiempo de ejecución: 0.031344 [s]
    \end{itemize}
\end{description}

\begin{figure}[H]
  \centering
  \includegraphics[width=0.666\textwidth]{img/DBScan.png}
  \caption{DBScan}
\end{figure}

\newpage





\subsection*{1.6. \; C-Means}

\begin{description}
  \item[Algoritmo:] C-Means
  \item[Parámetros utilizados:] \hfill
    \begin{itemize}
      \item c: 3
      \item m: 0.01
      \item error: 0.3
      \item maxiter: 20
      \item seed: None
    \end{itemize}
  \item[Resultados]\hfill
    \begin{itemize}
      \item Errores: 10-50\footnote{Muy variable}
      \item Tiempo de ejecución: 0.017491 [s]
    \end{itemize}
\end{description}

\begin{figure}[H]
  \centering
  \includegraphics[width=0.666\textwidth]{img/FuzzyCMeans.png}
  \caption{C-Means}
\end{figure}

\newpage




\subsection*{1.7. \; Mean shift}

\begin{description}
  \item[Algoritmo:] Mean shift
  \item[Parámetros utilizados:] \hfill
    \begin{itemize}
      \item bandwidth: 0.9
      \item bin\_seeding: True
      \item min\_bin\_freq: 1
      \item cluster\_all: False
    \end{itemize}
  \item[Resultados]\hfill
    \begin{itemize}
      \item Errores: 33
      \item Tiempo de ejecución: 0.080781 [s]
    \end{itemize}
\end{description}

\begin{figure}[H]
  \centering
  \includegraphics[width=0.666\textwidth]{img/Meanshift.png}
  \caption{Mean shift}
\end{figure}

\newpage




\subsection*{1.8. \; Spectral Clustering}

\begin{description}
  \item[Algoritmo:] Spectral Clustering
  \item[Parámetros utilizados:] \hfill
    \begin{itemize}
      \item n\_clusters: 3
      \item n\_components: 3
      \item eigen\_solver: ‘arpack’
      \item assign\_labels: ‘discretize’
      \item n\_init: 1
      \item weight: 9.5
    \end{itemize}
  \item[Resultados]\hfill
    \begin{itemize}
      \item Errores: 6
      \item Tiempo de ejecución: 0.064943 [s]
    \end{itemize}
\end{description}

\begin{figure}[H]
  \centering
  \includegraphics[width=0.666\textwidth]{img/SpectralClustering.png}
  \caption{Spectral Clustering}
\end{figure}

\newpage


\subsection*{2 \; Tabla comparativa}
En la siguiente tabla se resumen los errores y tiempos de cada algoritmo
\begin{center}
  \begin{tabular*}{0.666\textwidth}{@{\extracolsep{\fill}}l l l@{}}
      \midrule[1pt]
      Algoritmo & Errores & Tiempo [s] \\
      \midrule[0.4pt]
      Spectral Clustering  & 6     & 0.064943  \\
      k-means              & 14    & 0.012385  \\
      Minibatch k-means    & 15    & 0.016015  \\
      HAC                  & 16    & 0.002657  \\
      Ward                 & 16    & 0.011104  \\
      DBScan               & 22    & 0.031344  \\
      Mean shift           & 33    & 0.080781  \\
      C-Means              & 10-50 & 0.017491  \\
      \midrule[0.4pt]
  \end{tabular*}
\end{center}


\section*{Conclusiones}

\lipsum[3]

\section*{Referencias}

  \begin{itemize}
    \item ---
  \end{itemize}

\section*{Anexo}

  % \python{Tarea 1}{Implementación de los algoritmos}
  % \python{../Codigos/p3}{Pregunta 3. }

\end{document}
